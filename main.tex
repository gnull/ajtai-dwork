\documentclass[oneside, a4paper]{article}

\usepackage[T1,T2A]{fontenc}
\usepackage[utf8]{inputenc}
\usepackage[english,russian]{babel}
\usepackage{url, hyperref}
\usepackage[shortlabels]{enumitem}
\usepackage{amssymb,amsthm,amsmath,mathtools}
\usepackage{listings}

\usepackage{geometry}
 \geometry{
   a4paper,
   left=20mm,
   right=20mm,
 }

\usepackage{filecontents}
\begin{filecontents}{main.bib}
@article{ajtaidwork,
  author    = {Mikl{\'{o}}s Ajtai and
               Cynthia Dwork},
  title     = {The First and Fourth Public-Key Cryptosystems with Worst-Case/Average-Case
               Equivalence.},
  journal   = {Electronic Colloquium on Computational Complexity {(ECCC)}},
  volume    = {14},
  number    = {097},
  year      = {2007},
  url       = {http://eccc.hpi-web.de/eccc-reports/2007/TR07-097/index.html},
  timestamp = {Tue, 14 Aug 2018 17:08:03 +0200},
  biburl    = {https://dblp.org/rec/bib/journals/eccc/AjtaiD07},
  bibsource = {dblp computer science bibliography, https://dblp.org}
}
\end{filecontents}

\newtheorem{thm}{Утверждение}
\makeatletter
\@addtoreset{thm}{section}
\makeatother

\renewcommand{\thesection}{}
\renewcommand{\thesubsection}{}

\DeclareMathOperator{\lcm}{LCM}
\DeclareMathOperator*{\argmax}{arg\,max}

\DeclarePairedDelimiter\abs{\lvert}{\rvert}
\DeclarePairedDelimiter\ang{\langle}{\rangle}

\DeclareRobustCommand{\divby}{%
  \mathrel{\vbox{\baselineskip.65ex\lineskiplimit0pt\hbox{.}\hbox{.}\hbox{.}}}%
}

\begin{document}

\title{Конспект статьи \\ \foreignlanguage{english}{<<The First and Fourth Public-Key Cryptosystems
with Worst-Case/Average-Case Equivalence>>  \cite{ajtaidwork}}}
\author{Олейников Иван \\ \url{ivan.oleynikov95@gmail.com}}
\date{\today}
\maketitle

\section{Введение}

BibTeX provides for the storage of all references in an external, flat-file
database. (BibLaTeX uses this same syntax.) This database can be referenced in
any LaTeX document, and citations made to any record that is contained within
the file. This is often more convenient than embedding them at the end of every
document written; a centralized bibliography source can be linked to as many
documents as desired (write once, read many!). Of course, bibliographies can
be split over as many files as one wishes, so there can be a file containing
sources concerning topic A (a.bib) and another concerning topic B (b.bib). When
writing about topic AB, both of these files can be linked into the document
(perhaps in addition to sources ab.bib specific to topic AB).

\bibliographystyle{plain}
\foreignlanguage{english}{
  \bibliography{main}
}

\end{document}
