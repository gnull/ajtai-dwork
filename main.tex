\documentclass[oneside, a4paper]{article}

\usepackage[T1,T2A]{fontenc}
\usepackage[utf8]{inputenc}
\usepackage[english,russian]{babel}
\usepackage{url, hyperref}
\usepackage[shortlabels]{enumitem}
\usepackage{amssymb,amsthm,amsmath,mathtools}
\usepackage{listings}

\usepackage{geometry}
 \geometry{
   a4paper,
   left=20mm,
   right=20mm,
 }

\usepackage{filecontents}

\theoremstyle{plain}
\newtheorem{theorem}{Теорема}
\newtheorem{lemma}[theorem]{Лемма}
\theoremstyle{definition}
\newtheorem{definition}{Определение}
\newtheorem{example}{Пример}
\theoremstyle{remark}
\newtheorem{remark}{Замечание}

% \renewcommand{\thesection}{}
% \renewcommand{\thesubsection}{}

\DeclareMathOperator{\lcm}{LCM}

\newcommand\p{\ensuremath \mathbf p}
\newcommand\N{\ensuremath \mathcal N}
\newcommand\R{\ensuremath \mathbb R}

\DeclarePairedDelimiter\abs{\lvert}{\rvert}
\DeclarePairedDelimiter\ang{\langle}{\rangle}

\DeclareRobustCommand{\divby}{%
  \mathrel{\vbox{\baselineskip.65ex\lineskiplimit0pt\hbox{.}\hbox{.}\hbox{.}}}%
}

\begin{document}

\title{Конспект статьи \\ \foreignlanguage{english}{<<The First and Fourth Public-Key Cryptosystems
with Worst-Case/Average-Case Equivalence>>  \cite{ajtaidwork}}}
\author{Олейников Иван \\ \url{ivan.oleynikov95@gmail.com}}
\date{\today}
\maketitle

Этот документ содержит конспект того, что автор планировал рассказать за одно
занятие на семинара по теоретической информатике в СПбГУ
\footnote{\url{https://groups.google.com/forum/\#!forum/spbsu-teorseminar}}.
Секции << \dots >> были разобраны на семинаре устно, поэтому описаны очень
кратко, с расчётом на посетившего семинар читателя, к тому же они составляют
самую простую часть статьи и их нетрудно изучить самостоятельно. Остальные
секции << \dots >> содержат наиболее трудную для понимания часть статьи~---
докладчику не удалось разобрать их к началу семинара, поэтому после семинара был
написан этот конспект.

% В статье \cite{ajtaidwork} Ajtai и Dwork предлагают улучшенную версию своей
% первой криптосистемы, основанной на трудности решения в худшем случае задачи
% \foreignlanguage{english}{$f(n)$-unique Shortest Vector}~--- сокращённо uSVP.
% Дополнительно в статье рассматривается и первая криптосистема, 

\tableofcontents

\section{Базовые определения}



\bibliographystyle{plain}
\foreignlanguage{english}{
  \bibliography{main}
}
\begin{filecontents}{main.bib}
@article{ajtaidwork,
  author    = {Mikl{\'{o}}s Ajtai and
               Cynthia Dwork},
  title     = {The First and Fourth Public-Key Cryptosystems with Worst-Case/Average-Case
               Equivalence.},
  journal   = {Electronic Colloquium on Computational Complexity {(ECCC)}},
  volume    = {14},
  number    = {097},
  year      = {2007},
  url       = {http://eccc.hpi-web.de/eccc-reports/2007/TR07-097/index.html},
  timestamp = {Tue, 14 Aug 2018 17:08:03 +0200},
  biburl    = {https://dblp.org/rec/bib/journals/eccc/AjtaiD07},
  bibsource = {dblp computer science bibliography, https://dblp.org}
}
\end{filecontents}

\end{document}
