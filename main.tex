\documentclass[oneside, a4paper]{article}

\usepackage[T1,T2A]{fontenc}
\usepackage[utf8]{inputenc}
\usepackage[english,russian]{babel}
\usepackage{url, hyperref}
\usepackage[shortlabels]{enumitem}
\usepackage{amssymb,amsthm,amsmath,mathtools}
\usepackage{listings}
\usepackage[many]{tcolorbox}

\usepackage{geometry}
 \geometry{
   a4paper,
   left=20mm,
   right=20mm,
 }

\usepackage{filecontents}

\theoremstyle{plain}
\newtheorem{theorem}{Теорема}
\newtheorem{lemma}[theorem]{Лемма}
\newtheorem{exce}{Упражнение}
\theoremstyle{defn}
\newtheorem{defn}{Определение}
\newtheorem{example}{Пример}
\theoremstyle{remark}
\newtheorem{remark}{Замечание}

% \renewcommand{\thesection}{}
% \renewcommand{\thesubsection}{}

\newenvironment{definition}[1]{%
\begin{tcolorbox}[breakable,enhanced]
\begin{defn}[#1]
}{%
\end{defn}
\end{tcolorbox}
}

\newenvironment{greybox}[1]{%
\begin{tcolorbox}[breakable,enhanced, adjusted title = #1]
}{%
\end{tcolorbox}
}

\DeclareMathOperator{\lcm}{LCM}
\DeclareMathOperator{\round}{Round}
\DeclareMathOperator{\Span}{span}

\newcommand\p{\ensuremath \mathbf p}
\newcommand\N{\ensuremath \mathcal N}
\newcommand\PP{\ensuremath \mathcal P}
\newcommand\R{\ensuremath \mathbb R}
\newcommand\Z{\ensuremath \mathbb Z}

\DeclarePairedDelimiter\abs{\lvert}{\rvert}
\DeclarePairedDelimiter\norm{\lVert}{\rVert}
\DeclarePairedDelimiter\ang{\langle}{\rangle}

\DeclareRobustCommand{\divby}{%
  \mathrel{\vbox{\baselineskip.65ex\lineskiplimit0pt\hbox{.}\hbox{.}\hbox{.}}}%
}

\begin{document}

\title{Конспект статьи \\ \foreignlanguage{english}{<<The First and Fourth Public-Key Cryptosystems
with Worst-Case/Average-Case Equivalence>>  \cite{ajtaidwork}}}
\author{Олейников Иван \\ \url{ivan.oleynikov95@gmail.com}}
\date{\today}
\maketitle

Этот документ содержит конспект того, что автор планировал рассказать за одно
занятие на семинара по теоретической информатике в СПбГУ
\footnote{\url{https://groups.google.com/forum/\#!forum/spbsu-teorseminar}}.
Секции << \dots >> были разобраны на семинаре устно, поэтому описаны очень
кратко, с расчётом на посетившего семинар читателя, к тому же они составляют
самую простую часть статьи и их нетрудно изучить самостоятельно. Остальные
секции << \dots >> содержат наиболее трудную для понимания часть статьи~---
докладчику не удалось разобрать их к началу семинара, поэтому после семинара был
написан этот конспект.

В статье \cite{ajtaidwork} Ajtai и Dwork предлагают улучшенную версию (они её
называют иногда \foreignlanguage{english}{<<The Compact Cryptosystem>>} или
\foreignlanguage{english}{<<Fourth Cryptosystem>>}) своей
первой криптосистемы, основанной на трудности решения в худшем случае задачи
\foreignlanguage{english}{$f(n)$-unique Shortest Vector}.
Дополнительно в статье рассматривается и первая криптосистема, но в этом
конспекте мы о ней говорить не будем.

\tableofcontents

\section{Предварительные определения}

\begin{greybox}{Криптосистема}
Криптосистема, которую мы опишем в этом конспекте, имеет такие параметры:
\begin{description}
\item[$n$]~--- параметр надёжности.
\item[$l$]~--- длина шифруемого сообщения в битах. Иногда мы будем
пользоваться для удобства сокращением $m = n + l$.
\end{description}

\tcblower

И будет работать с данными в таком формате (вектора имеются ввиду из $\R^m$):
\begin{description}
\item[Секретный ключ~---] набор взаимно ортогональных векторов $\{u_0 \dots
u_l\}$. Мы будем пользоваться сокращениями $e_{u_i} = u_i / \norm{u_i}$ и $z_i =
e_{u_i} / \norm{u_i}$.
\item[Публичный ключ~---] это
\begin{itemize}
\item $V = \{v_0 \dots v_l\}$~--- набор векторов,
\item $D = \{d_1 \dots d_{m'}\}$~--- набор векторов и
\item $\displaystyle \PP = \biggl\{\sum_{i=1}^m \alpha_i p_i \mid
\alpha_1, \alpha_2 \dots \alpha_m \in [0, 1) \biggr\}$~--- параллелепипед,
задаваемый набором линейно независимых векторов $p_1 \dots p_m$.
\end{itemize}
\item[Шифруемое сообщение~---] строка $b_0 b_1 \dots b_l$ из $l+1$ бит.
\item[Шифротекст~---] вектор $x$.
\end{description}
\end{greybox}

\begin{description}
\item[$\round_\alpha(x)$] $= i\alpha$, где $i = \max \{ i' \in \Z \mid i' \alpha
\leq x\}$. То есть это $x$ округлённое вниз до ближайшего кратного $\alpha$.
Если $x$~--- вектор, то будем округлять каждую его координату. Когда мы будем
выбирать какие-то значения из распределений на вещественных числах, мы всегда
будем округлять их до ближайших кратных $2^{-\p}$.
\item[$\N_m(0, \sigma^2)$] --- нормальное распределение на векторах из $\R^m$.
Вектор из этого распределения получается выбором каждой его координаты
независимо из одномерного нормального распределения с мат. ожиданием в нуле и
среднеквадратичным отклонением $\sigma$~--- $\N(0, \sigma^2)$.
\end{description}

\section{Описание криптосистемы}

% «Сущности»? WTF?
Напомним, какой тип имеют сущности, которыми оперирует криптосистема, и какими
символами мы их обозначаем.

\begin{table}[h]
\centering
\begin{tabular}{c | c | c}
Сущность & Обозначение & Формат \\
\hline
Секретный ключ & $\{u_0 \dots u_l\}$ & Набор из $l+1$ ортогональных векторов \\
Публичный ключ
&
\parbox{3cm}{
\[
\begin{aligned}
\PP &= \{p_1 \dots p_m\} \\
V &= \{v_0 \dots v_l\} \\
D &= \{d_1 \dots d_{m'}\}
\end{aligned}
\]
}
& \dots \\
Шифруемое сообщение & $b_0 \dots b_l$ & Строка из $l+1$ бит \\
Код сообщения & $x$ & Внутренняя точка параллелепипеда $\PP^-$ \\
\end{tabular}
\end{table}

Напомним, что публичный ключ~--- это тройка $pk = (\PP, V, D)$, где $\PP$~---
$m$-мерный параллелепипед, $V$~--- набор из $l+1$ векторов, $D$~--- набор из
$m'$ векторов. А секретный ключ~--- это набор векторов $\{u_0 \dots u_l\}$. Все
упомянутые выше вектора лежат в $\R^m$.

\subsection{Генерация ключей}

\subsection{Шифрование} Шифрующий алгоритм принимает на вход сообщение $b_0 b_1 \dots
b_l$ из $l+1$ бит. Выберем равномерно и независимо $m'$ бит $\delta_1 \dots
\delta_{m'}$, и вернём в качестве результата шифрования значение
\[
x = \sum_{i = 0}^l b_i v_i + \sum_{i = 1}^m 2 \delta_i d_i \mod \PP.
\]

Первое слагаемое в этой сумме~--- это код сообщения, по нему можно восстановить
сообщение, даже не имея секретного ключа, что демонстрируется упражнением ниже.
А второе слагаемое~--- это <<шум>>, который накладывается, чтобы не
имеющий секретного ключа не мог восстановить данные, а имеющий секретный
ключ мог это сделать, избавившись от шума. То есть, секретный ключ позволит нам
каким-то образом выделить $\sum b_i v_i \mod \PP$ из суммы выше.

\begin{exce}
По значению
\[
\sum_{i=0}^l b_i v_i \mod \PP
\]
можно восстановить $b_0 \dots b_l$.
\end{exce}

\subsection{Дешифрование} Дешифрующий алгоритм, которому известен секретный ключ
$\{u_0 \dots u_l\}$ и открытый ключ $(\PP, V, D)$, получает на вход $x$,
выданный шифрующим алгоритмом из абзаца выше, и восстанавшивает оригинальное
сообщение $b_0 \dots b_l$.

\subsection{Доказательство корректности}
Докажем, что зашифрованные такой криптосистемой данные будут корректно
восстановлены с пренебрежимо малой вероятностью ошибки.

\section{Доказательство надёжности}
Сведём решение \foreignlanguage{english}{worst-case} задачи
\foreignlanguage{english}{Unique Shortest Vector} к взлому определённой нами
криптосистемы.

\bibliographystyle{plain}
\foreignlanguage{english}{
  \bibliography{main}
}
\begin{filecontents}{main.bib}
@article{ajtaidwork,
  author    = {Mikl{\'{o}}s Ajtai and
               Cynthia Dwork},
  title     = {The First and Fourth Public-Key Cryptosystems with Worst-Case/Average-Case
               Equivalence.},
  journal   = {Electronic Colloquium on Computational Complexity {(ECCC)}},
  volume    = {14},
  number    = {097},
  year      = {2007},
  url       = {http://eccc.hpi-web.de/eccc-reports/2007/TR07-097/index.html},
  timestamp = {Tue, 14 Aug 2018 17:08:03 +0200},
  biburl    = {https://dblp.org/rec/bib/journals/eccc/AjtaiD07},
  bibsource = {dblp computer science bibliography, https://dblp.org}
}
\end{filecontents}

\end{document}
