\documentclass[oneside, a4paper]{article}

\usepackage[T1,T2A]{fontenc}
\usepackage[utf8]{inputenc}
\usepackage[english,russian]{babel}
\usepackage{url, hyperref}
\usepackage[shortlabels]{enumitem}
\usepackage{amssymb,amsthm,amsmath,mathtools}
\usepackage{listings}

\usepackage{geometry}
 \geometry{
   a4paper,
   left=20mm,
   right=20mm,
 }

\usepackage{filecontents}

\theoremstyle{plain}
\newtheorem{theorem}{Теорема}
\newtheorem{lemma}[theorem]{Лемма}
\theoremstyle{definition}
\newtheorem{definition}{Определение}
\newtheorem{example}{Пример}
\theoremstyle{remark}
\newtheorem{remark}{Замечание}

% \renewcommand{\thesection}{}
% \renewcommand{\thesubsection}{}

\DeclareMathOperator{\lcm}{LCM}

\newcommand\p{\ensuremath \mathbf p}
\newcommand\N{\ensuremath \mathcal N}
\newcommand\R{\ensuremath \mathbb R}

\DeclarePairedDelimiter\abs{\lvert}{\rvert}
\DeclarePairedDelimiter\ang{\langle}{\rangle}

\DeclareRobustCommand{\divby}{%
  \mathrel{\vbox{\baselineskip.65ex\lineskiplimit0pt\hbox{.}\hbox{.}\hbox{.}}}%
}

\begin{document}

\title{Конспект статьи \\ \foreignlanguage{english}{<<The First and Fourth Public-Key Cryptosystems
with Worst-Case/Average-Case Equivalence>>  \cite{ajtaidwork}}}
\author{Олейников Иван \\ \url{ivan.oleynikov95@gmail.com}}
\date{\today}
\maketitle

Этот документ содержит конспект того, что автор планировал рассказать за одно
занятие на семинара по теоретической информатике в СПбГУ
\footnote{\url{https://groups.google.com/forum/\#!forum/spbsu-teorseminar}}.
Секции << \dots >> были разобраны на семинаре устно, поэтому описаны очень
кратко, с расчётом на посетившего семинар читателя, к тому же они составляют
самую простую часть статьи и их нетрудно изучить самостоятельно. Остальные
секции << \dots >> содержат наиболее трудную для понимания часть статьи~---
докладчику не удалось разобрать их к началу семинара, поэтому после семинара был
написан этот конспект.

% В статье \cite{ajtaidwork} Ajtai и Dwork предлагают улучшенную версию своей
% первой криптосистемы, основанной на трудности решения в худшем случае задачи
% \foreignlanguage{english}{$f(n)$-unique Shortest Vector}~--- сокращённо uSVP.
% Дополнительно в статье рассматривается и первая криптосистема, 

\tableofcontents

\section{Базовые определения}

\begin{definition}[Криптосистема с открытым ключом]
Криптосистема с открытым ключом~--- это тройка полиномиальных по времени
вероятностных алгоритмов $(G, E, D)$. Алгоритм $G$ используется для генерации
пары ключей: публичного $pk$ и секретного $sk$.
Алгоритм $E$, получив на вход публичный ключ и сообщение $m$, шифрует сообщение
и выдаёт шифротекст $c$. Имея секретный ключ, можно
расшифровать сообщение, зашифрованное парным ему открытым ключом:
\[
\begin{aligned}
(pk, sk) &\gets G(1^n), \\
c        &\gets E(pk, m), \\
m'       &\gets D(sk, c). \\
\end{aligned}
\]

Мы будем требовать, чтобы вероятность корректного восстановления сообщения была
$> 2/3$ при достаточно больших $n$. Значение $n$~--- это параметр надёжности,
его смысл станет ясен в следующем определении. (Ещё для удобства будем считать,
что по $pk$ и $sk$ можно за полином вычислить $1^n$, ведь если это не так,
можно просто заставить $G$ приписать $1^n$ к обоим ключам.)
\end{definition}

\begin{definition}[Надёжность криптосистемы]
Пусть $A$~--- это некоторый вероятностный полиномиальный по времени алгоритм,
который мы будем называть противником, а $(G, E, D)$~--- криптосистема.
Рассмотрим такой вероятностный эксперимент:
\[
\begin{aligned}
(pk, sk) &\gets G(1^n)             &  \\
(m_0, m_1)  &\gets A(pk)           & &\text{ ($A$ выбирает два разных сообщения)} \\
i        &\gets \{0, 1\}           &  \\
c        &\gets E(pk, m_i)         & &\text{ ($E$ шифрует случаное из них)} \\
i'       &\gets A(pk, m_0, m_1, c) & &\text{ ($A$ должен угадать, какое было
зашифровано)} \\
\end{aligned}
\]
Криптосистема $(G, E, D)$ называется надёжной, если вероятность того, что в
результате такого вероятностного эксперимента окажется $i = i'$ и $m_0 \neq
m_1$, пренебрежимо малая функция от $n$. Вероятность берётся по случайным битам
всех трёх алгоритмов $G$, $A$ и $E$ и по случайному биту $i$.

Функция $f(n)$ называется пренебрежимо малой, если для любого полинома $p(n)$
$f(n) < p(n)$ при достаточно больших $n$.
\end{definition}

\bibliographystyle{plain}
\foreignlanguage{english}{
  \bibliography{main}
}
\begin{filecontents}{main.bib}
@article{ajtaidwork,
  author    = {Mikl{\'{o}}s Ajtai and
               Cynthia Dwork},
  title     = {The First and Fourth Public-Key Cryptosystems with Worst-Case/Average-Case
               Equivalence.},
  journal   = {Electronic Colloquium on Computational Complexity {(ECCC)}},
  volume    = {14},
  number    = {097},
  year      = {2007},
  url       = {http://eccc.hpi-web.de/eccc-reports/2007/TR07-097/index.html},
  timestamp = {Tue, 14 Aug 2018 17:08:03 +0200},
  biburl    = {https://dblp.org/rec/bib/journals/eccc/AjtaiD07},
  bibsource = {dblp computer science bibliography, https://dblp.org}
}
\end{filecontents}

\end{document}
